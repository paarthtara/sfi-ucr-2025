\documentclass[10pt]{article}        	%sets the font to 12 pt and says this is an article (as opposed to book or other documents)
\usepackage{amsfonts, amssymb, bbm, amsmath}	
\usepackage{braket, graphicx, lipsum, geometry}
\usepackage{mathrsfs, setspace}

\setstretch{1.1}

%\usepackage[light]{kurier}
%\usepackage[light]{CormorantGaramond}
%\usepackage{boisik}
\usepackage{lmodern}
\usepackage[T1]{fontenc}
% packages to get the fonts, symbols used in most math
  
%\usepackage{setspace}               		% Together with \doublespacing below allows for doublespacing of the document

\oddsidemargin=-0.5cm                 	% These three commands create the margins required for class
\geometry{margin=1.25in}
%\setlength{\textwidth}{6.5in}         	%
\addtolength{\voffset}{-20pt}        		%
\addtolength{\headsep}{25pt}           	%
\pagestyle{myheadings}                           	% tells LaTeX to allow you to enter information in the heading
\markright{Paarth Tara \hfill} 	% put your name instead of Murphy Waggoner 

\newcommand{\eqn}[0]{\begin{array}{rcl}}%begin an aligned equation - allows for aligning = or inequalities.  Always use with $$ $$
\newcommand{\eqnend}[0]{\end{array} }  	%end the aligned equation

\newcommand{\qed}[0]{$\square$}        	% make an unfilled square the default for ending a proof

\newcommand{\ketbra}[2]{\ket{#1}\bra{#2}}
\newcommand{\curlyparen}[1]{\left\{#1\right\}}
\newcommand{\trace}[1]{\text{tr}\{#1\}}
\newcommand{\Prob}[1]{\text{Prob}(#1)}
\newcommand{\defeq}{\stackrel{\text{def}}=}
\newcommand{\vecii}[2]{\begin{pmatrix}
	#1 \\ #2
\end{pmatrix}}
\newcommand{\paren}[1]{\left(#1\right)}
\newcommand{\parfrac}[2]{\paren{\frac{#1}{#2}}}

\newcommand{\problem}[3]{{\noindent \textbf{#1.} \textit{#2}
\newcommand{\ts}{\textsuperscript} % places text above

\smallskip
\noindent \textsc{Solution}. #3}}

%\doublespacing                         	% Together with the package setspace above allows for doublespacing of the document

\begin{document}

%%%%%%%%%%%%%%%%%%%%%%%%%%%% HEAD %%%%%%%%%%%%%%%%%%%%%%%%%
\begin{center}
	\large{\textsc{SFI Research --- Notes 1}}\\\vspace{6pt}
	\small{\text{Summer 2025}}
\end{center}
%\noindent\rule{\textwidth}{0.3pt}
\vspace{7pt}
%%%%%%%%%%%%%%%%%%%%%%%%%%%% HEAD %%%%%%%%%%%%%%%%%%%%%%%%%%

%%%%%%%%%%%%%%%%%%%%%%%%%%%% BODY %%%%%%%%%%%%%%%%%%%%%%%%%%

\noindent \textbf{Problem 1.} We have a game that starts with $k$ dots assorted into a row. $$\circ\circ\cdots \circ\circ$$ The game proceeds as follows. Each dot is \textit{movable} if there exists a stack of dots adjacent to it. That is, a dot is \textit{not movable} only if it is not directly surrounded by any dots. For each time step, a random movable dot is moved to one of the adjacent positions at random. The game proceeds until isolated \textit{stacks} have been formed, at which point the game stalls.

The question in this game is the nature of the final state of the system. Many questions can be asked, such as:
\begin{itemize}
	\item What is the average stack length at the end of the game?
	\item What is the average number of stacks, or the average number of gaps or gap length?
\end{itemize}

\noindent This problem encapsulates some elements of the slow-relaxation and frustration processes seen in glassy systems, and is one of many \textit{toy models} made to get a sense of these slow-relaxation dynamics.

\bigskip
\noindent \textbf{Example Game.} Consider a system of 6 dots. Here's an example evolution\footnote{$\stackrel{k}{\circ}$ denotes a stack of $k$ circles.}: $$\circ\circ\circ\circ\circ\,\circ \rightarrow \_\stackrel{2}{\circ}\circ\circ\circ\,\circ\rightarrow \_\stackrel{3}{\circ}\_\circ\circ\,\circ\rightarrow \_\stackrel{3}{\circ}\_\_\stackrel{2}{\circ}\circ\rightarrow \_\stackrel{3}{\circ}\_\_\circ\stackrel{2}{\circ}\rightarrow \_\stackrel{3}{\circ}\_\_\_\stackrel{3}{\circ}.$$ At this point, the final stacks cannot change. Note that there can be lots of flip-flopping between states, and the relaxation to the final state can be very slow, similar to what is seen in many glassy systems. This is a fundamental property of this game.

\bigskip
\noindent \textbf{Notes.} I had failed to really think about this aspect of the game before, the slow-relaxation. A certain final configuration can be achieved in a number of ways, for various scales of time. For example, there can be nearly infinite flip-flopping before the stacks finally settle to a certain pattern.

For this reason, even a computerized enumeration of this game would be insufficient for a full analysis (given the algorithm is naive enough). There are two important questions, more,

\begin{itemize}
	\item Do we need to consider the \textit{very} long-term behavior of the game to get an idea of it? (If so, then this game becomes way more interesting).
	\item Is it useful to decompose the final states as those that are achievable at some time $T$, and to vary $T$ to get increasingly greater ``resolution'' of the final state dynamics?
\end{itemize}

\bigskip
\noindent \textbf{Proposition.} For a $k$-lattice game (not on a ring), the number of final possible arrangements of the game is (at least) given by the bound $$\#(\text{final states of }k\text{-lattice})\leq \sum_{s=1}^{\lceil k/2\rceil}{k-s+1\choose s}{k-1\choose s-1}.$$

\bigskip
\noindent \textit{Proof.} First, let us consider this problem by considering how many \textit{stacks} will be in a final arrangement. Clearly, the final arrangement will be of the form of isolated stacks, as seen in the previous example game. At most, there can be at most $\lceil k/2\rceil$ stacks. Let $s$ denote the number of stacks in a final arrangement, and let us count the number of ways the final system can achieve a configuration of $s$ stacks.

First, let us count the number of ways we can achieve $s$ isolated stacks on a $k$-lattice. ``Isolated'' means that each stack is separated by another stack by at least one empty gap. Let us put in these $s-1$ gaps to the right of each of the first $s-1$ stacks implicitly. We remove these gaps from our consideration, and are left with $k-(s-1)=k-s+1$ slots with $s$ stacks and remaining gaps. The way to arrange these is $${k-s+1\choose s},$$ and upon inserting back the implicit $s-1$ gaps, we are left with this number of unique stacking arrangements.

Now, for each stacking arrangement, let us count the number of ways to partition the $k$ balls between the $s$ stacks. Each stack has at least one ball, so automatically we are actually partitioning $k-s$ balls into $s$ stacks (with each stack having at least zero balls). We have $s-1$ dividers and $k-s$ balls, so a total of $(k-s)+(s-1)=k-1$ objects and $s-1$ dividers for a total of $${k-1\choose s-1}$$ arrangements. Hence, if every stacking arrangement is theoretically possible, then the number of possible unique final states is given by $$\sum_{s=1}^{\lceil k/2\rceil} {k-s+1\choose s}{k-1\choose s-1}.$$ This statement is not proven, so we have just a bound on the number of final states in a $k$-lattice, right now.


\bigskip
\noindent \textbf{Task.} Manually enumerate a number of games, and calculate their probabilities (accounting for the flip-flopping dynamics), and see if they have a simple nature for small $k$. Proceed to write both (1) an enumeration and (2) a simulation of the game to see if it can provide some insight into large $k$.

%%%%%%%%%%%%%%%%%%%%%%%%%%%% BODY %%%%%%%%%%%%%%%%%%%%%%%%%%

\end{document}

%%%%%%%%%%%%%%%%%%%%%%%%%%%% EXAMPLE %%%%%%%%%%%%%%%%%%%%%%%
%\problem{1}{Problem statement.}{\lipsum[42]}{0.55cm}
%
%\bigskip
%\problem{2}{Problem statement.}{
%	\problem{a}{Sub-problem 1}{\lipsum[42]}{1.1cm}
%}{0.55cm}

























